\documentclass[pdf]{beamer}
\mode<presentation>{}

\usepackage[utf8]{inputenc}

\usepackage[sans]{dsfont}
\usepackage{amsmath}
\usepackage{gensymb}
\usepackage{mathtools}
\usepackage{eqparbox}

\newcommand{\op}[1]{\operatorname{#1}}
\newcommand{\bbf}[1]{\mathds{#1}}
\newcommand{\Z}{\bbf{Z}}

\title{Bernsteinrelationen, der $2$-Kozykel $\bbf{X}$, und der Raum der Orientierungen}
\author{Nicolas A. Schmidt}
\date{}

\begin{document}

%% title frame
\begin{frame}
   \titlepage
\end{frame}

\begin{frame}{Überblick}
\begin{itemize}
   \item<2-> Zentrales Objekt: \textbf{Hecke-Algebren} \pause[3](Iwahori-Hecke-Algebren, affine Hecke-Algebren, pro-$p$-Iwahori Hecke-Algebren, \dots)\pause[4], genauer \textbf{generische pro$\text{-}p$ Hecke-Algebren}
   \item<5-> Hecke Algebren $\mathcal{H}$ werden gebildet zu
      \begin{itemize}
         \item<6-> einer \textbf{Coxetergruppe}\pause[7] (bzw. einer ``pro-$p$ Coxetergruppe'')
         \item<8-> und einer \textbf{Familie von Parametern}
      \end{itemize}
   \item<9-> Zentrales Ziel: Bestimmung der Struktur von $\mathcal{H}$\pause[10], insbesondere Bestimmung des \textbf{Zentrums} $Z(\mathcal{H})$
      \pause[11]\[ Z(\mathcal{H}) = \{ z \in \mathcal{H}\ :\ zx = xz\text{ für alle } x \in \mathcal{H} \} \]
   \item<12-> Ideologie:
      \begin{itemize}
         \item<13-> Coxetergruppen sind \textit{geometrische} Objekte
         \item<14-> Verstehe $\mathcal{H}$ vermöge dieser Geometrie!
      \end{itemize}
\end{itemize}
\end{frame}

\begin{frame}{Das einfachste mathematische Objekt}
   \pause[3]
   \begin{tabular}[t]{rl}
      \textbf{Frage:}& Was ist die Gruppe der \textit{Symmetrien} von $\Delta$? \\
      \uncover<24->{\textbf{Antw.:}}& \pause[4]$W = \only<4-6>{\text{?}}\only<7-9>{\{ s, t, u, ... \}}\only<10>{\{ s, t, sts, ... \}}\only<11-13>{\{ s, t, sts, st, ... \}}\only<14-17>{\{ s, t, sts, st, \temporal<15>{\eqmakebox[BOX1][l]{$(st)^2$}}{\eqmakebox[BOX1][l]{$(st)^{-1}$}}{\eqmakebox[BOX1][l]{$ts$}}, \alt<-16>{\eqmakebox[BOX3][l]{$(st)^3$}}{\eqmakebox[BOX3][l]{$1$}}, ... \}}\only<18-22>{\eqmakebox[BOX2][l]{$\{ s, t, sts, st, ts, 1, ... \}$}}\only<23->{\eqmakebox[BOX2][l]{$\{ s, t, sts, st, ts, 1 \}$}}\only<22>{\eqmakebox[BOX4][r]{$\quad \subseteq\quad S_3$}}\only<23>{\eqmakebox[BOX4][r]{$ \quad =\quad S_3$}}$ \\
      \only<12-18,26-28>{\textbf{NB:}} & \only<-18>{\uncover<12-18>{$st = \text{\textit{Drehung} um 120\degree} \pause[13]\quad\Rightarrow\quad \op{ord}(st) = 3$}}\only<26-28>{$W$ erhält $\mathfrak{H} = \{ s, t, u \}$\only<28->{ und \textit{Kammern}}}
   \end{tabular}
   \pause[2]
   \begin{figure}
   \centering%
      \includegraphics<2-4>[width=0.5\textwidth]{graphics/triangle.eps}%
      \includegraphics<5>[width=0.5\textwidth]{graphics/triangle2.eps}%
      \includegraphics<6-7>[width=0.5\textwidth]{graphics/triangle3.eps}%
      \includegraphics<8>[width=0.5\textwidth]{graphics/triangle4.eps}%
      \includegraphics<9-19>[width=0.5\textwidth]{graphics/triangle5.eps}%
      \includegraphics<20>[width=0.5\textwidth]{graphics/triangle6.eps}%
      \includegraphics<21-23>[width=0.5\textwidth]{graphics/triangle7.eps}%
      \includegraphics<24>[width=0.5\textwidth]{graphics/triangle8.eps}%
      \includegraphics<25-26>[width=0.5\textwidth]{graphics/triangle9.eps}%
      \includegraphics<27-30>[width=0.5\textwidth]{graphics/triangle10.eps}%
      \includegraphics<35->[width=0.5\textwidth]{graphics/triangle9.eps}%
   \end{figure}
\end{frame}

\begin{frame}{Geometrie der Spiegelungsgruppen}
\end{frame}

\end{document}
