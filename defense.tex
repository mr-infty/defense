\documentclass[pdf]{beamer}
\mode<presentation>{}
%
\usecolortheme{orchid}
%

\usepackage[utf8]{inputenc}

\usepackage[sans]{dsfont}
\usepackage{amsmath}
\usepackage{gensymb}
\usepackage{mathtools}
\usepackage{eqparbox}

\newcommand{\op}[1]{\operatorname{#1}}
\newcommand{\bbf}[1]{\mathds{#1}}
\newcommand{\Z}{\bbf{Z}}

\renewcommand*{\proofname}{Beweis}

\title{Bernsteinrelationen, der $2$-Kozykel $\bbf{X}$, und der Raum der Orientierungen}
\author{Nicolas A. Schmidt}
\date{}

\begin{document}

%% title frame
\begin{frame}
   \titlepage
\end{frame}

\section{Einführung}
\begin{frame}{Überblick}
\begin{itemize}
   \item<2-> Zentrales Objekt: \textbf{Hecke-Algebren} \pause[3](Iwahori-Hecke-Algebren, affine Hecke-Algebren, pro-$p$-Iwahori Hecke-Algebren, \dots)
   \item<4->Genauer: \textbf{generische pro$\text{-}p$ Hecke-Algebren}
   \item<5-> Hecke Algebren $\mathcal{H}$ werden gebildet zu
      \begin{itemize}
         \item<6-> einer \textbf{Coxetergruppe}\pause[7] (bzw. einer ``pro-$p$ Coxetergruppe'')
         \item<8-> und einer \textbf{Familie von Parametern}
      \end{itemize}
   \item<9-> Zentrales Ziel: Bestimmung der Struktur von $\mathcal{H}$\pause[10], insbesondere Bestimmung des \textbf{Zentrums} $Z(\mathcal{H})$
      \pause[11]\[ Z(\mathcal{H}) = \{ z \in \mathcal{H}\ :\ zx = xz\text{ für alle } x \in \mathcal{H} \} \]
   \item<12-> Leitgedanke:
      \begin{itemize}
         \item<13-> Coxetergruppen sind \textit{geometrische} Objekte
         \item<14-> Verstehe $\mathcal{H}$ vermöge dieser Geometrie!
      \end{itemize}
\end{itemize}
\end{frame}

\begin{frame}{Das einfachste mathematische Objekt}
   \pause[3]
   \begin{tabular}[t]{rl}
      \textbf{Frage:}& Was ist die Gruppe der \textit{Symmetrien} von $\Delta$? \\
      \uncover<24->{\textbf{Antw.:}}& \pause[4]$W = \only<4-6>{\text{?}}\only<7-9>{\{ s, t, u, ... \}}\only<10>{\{ s, t, sts, ... \}}\only<11-13>{\{ s, t, sts, st, ... \}}\only<14-17>{\{ s, t, sts, st, \temporal<15>{\eqmakebox[BOX1][l]{$(st)^2$}}{\eqmakebox[BOX1][l]{$(st)^{-1}$}}{\eqmakebox[BOX1][l]{$ts$}}, \alt<-16>{\eqmakebox[BOX3][l]{$(st)^3$}}{\eqmakebox[BOX3][l]{$1$}}, ... \}}\only<18-22>{\eqmakebox[BOX2][l]{$\{ s, t, sts, st, ts, 1, ... \}$}}\only<23->{\eqmakebox[BOX2][l]{$\{ s, t, sts, st, ts, 1 \}$}}\only<22>{\eqmakebox[BOX4][r]{$\quad \subseteq\quad S_3$}}\only<23>{\eqmakebox[BOX4][r]{$ \quad =\quad S_3$}}$ \\
      \only<12-18,26-28>{\textbf{NB:}}\only<29-30>{\textbf{Wähle}}\only<31-32>{\textbf{Zu}}\only<33-36>{\textbf{Finde}}\only<46-53>{\textbf{NB:}}\only<54-65>{\textbf{Fazit:}} & \only<-18>{\uncover<12-18>{$st = \text{\textit{Drehung} um 120\degree} \pause[13]\quad\Rightarrow\quad \op{ord}(st) = 3$}}\only<26-28>{$W$ erhält $\mathfrak{H} = \{ s, t, u \}$\only<28->{ und \textit{Kammern}}}\only<29-30>{\textit{Fundamentalkammer}\only<30>{ $C$}}\only<31-32>{ $w \in W$\only<32>{ \textbf{betrachte} $wC$}}\only<33-36>{\textit{Gallerie}}\only<34-36>{ $(C, \uncover<35->{D}, \uncover<36->{E}, wC)$ }\only<46-48,50-53>{ $stswC = C$ }\only<48>{$\Rightarrow$ $wC = stsC$ }\only<49>{$\Rightarrow$ Jede Kammer hat die Form $sts\dots C$}\only<51-53>{ $\Rightarrow stsw = 1$ }\only<52-53>{$\Leftrightarrow$ $w = sts$}\only<53>{ $\Rightarrow W = \left<s,t\right>$}\only<54-56>{$W\only<54>{ = \left<s,t\right>}$ \only<55-56>{wirkt einfach transitiv auf Kammern}}\only<57>{$W = \text{Kammern}$}\only<58-63>{Ausdruck \only<58>{\eqmakebox[BOXeql][l]{$1$}}\only<59-61>{\eqmakebox[BOXeql][l]{$\uncover<59->{s}\uncover<60->{t}\uncover<61->{s}$}}\only<62-63>{\eqmakebox[BOXeql][l]{$tst$}} $\widehat{=}$ Gallerie $(1\only<59-61>{, s}\only<62-63>{, t}\only<60-61>{, st}\only<62-63>{, ts}\only<61>{, sts}\only<62-63>{, tst})$}\only<64-65>{$sts = tst$ \uncover<65>{\quad (``Zopfrelation'')}}
   \end{tabular}
   \pause[2]
   \begin{figure}
   \centering%
      \includegraphics<2-4>[width=0.5\textwidth]{graphics/triangle.eps}%
      \includegraphics<5>[width=0.5\textwidth]{graphics/triangle2.eps}%
      \includegraphics<6-7>[width=0.5\textwidth]{graphics/triangle3.eps}%
      \includegraphics<8>[width=0.5\textwidth]{graphics/triangle4.eps}%
      \includegraphics<9-19>[width=0.5\textwidth]{graphics/triangle5.eps}%
      \includegraphics<20>[width=0.5\textwidth]{graphics/triangle6.eps}%
      \includegraphics<21-23>[width=0.5\textwidth]{graphics/triangle7.eps}%
      \includegraphics<24>[width=0.5\textwidth]{graphics/triangle8.eps}%
      \includegraphics<25-26>[width=0.5\textwidth]{graphics/triangle9.eps}%
      \includegraphics<27-28>[width=0.5\textwidth]{graphics/triangle10.eps}%
      \includegraphics<29>[width=0.5\textwidth]{graphics/triangle11.eps}%
      \includegraphics<30-31>[width=0.5\textwidth]{graphics/triangle12.eps}%
      \includegraphics<32-34>[width=0.5\textwidth]{graphics/triangle13.eps}%
      \includegraphics<35>[width=0.5\textwidth]{graphics/triangle14.eps}%
      \includegraphics<36-37>[width=0.5\textwidth]{graphics/triangle15.eps}%
      \includegraphics<38>[width=0.5\textwidth]{graphics/triangle16.eps}%
      \includegraphics<39>[width=0.5\textwidth]{graphics/triangle17.eps}%
      \includegraphics<40>[width=0.5\textwidth]{graphics/triangle18.eps}%
      \includegraphics<41>[width=0.5\textwidth]{graphics/triangle19.eps}%
      \includegraphics<42>[width=0.5\textwidth]{graphics/triangle20.eps}%
      \includegraphics<43>[width=0.5\textwidth]{graphics/triangle21.eps}%
      \includegraphics<44>[width=0.5\textwidth]{graphics/triangle22.eps}%
      \includegraphics<45-46>[width=0.5\textwidth]{graphics/triangle23.eps}%
      \includegraphics<47-53,55>[width=0.5\textwidth]{graphics/triangle12.eps}%
      \includegraphics<54>[width=0.5\textwidth]{graphics/triangle31.eps}%
      \includegraphics<56>[width=0.5\textwidth]{graphics/triangle24.eps}%
      \includegraphics<57-58>[width=0.5\textwidth]{graphics/triangle25.eps}%
      \includegraphics<59>[width=0.5\textwidth]{graphics/triangle26.eps}%
      \includegraphics<60>[width=0.5\textwidth]{graphics/triangle27.eps}%
      \includegraphics<61>[width=0.5\textwidth]{graphics/triangle28.eps}%
      \includegraphics<62>[width=0.5\textwidth]{graphics/triangle29.eps}%
      \includegraphics<63-65>[width=0.5\textwidth]{graphics/triangle30.eps}%
   \end{figure}
\end{frame}

\begin{frame}{Zusammenfassung}
   \begin{itemize}
      \item<1-> $W$ $\widehat{=}$ $\text{Kammern}$\pause , $1$ $\widehat{=}$ $C$
      \item<3-> $W = \left<S\right>$
      \item<4-> $S = \text{Spiegelungen an den Hyperebenen die $C$ berühren}$
      \item<5-> $\mathfrak{H} = \text{Spiegelungen in $W$} \pause[6]= \{ wsw^{-1} : w \in W, s \in S \}$
      \item<7-> Ausdrücke $s_1 s_2 s_3 \dots$ $\widehat{=}$ Gallerien $(1, s_1, s_1s_2, s_1s_2s_3, \dots)$
   \end{itemize}
\end{frame}

\section{Grundbegriffe}
\begin{frame}{Coxetergruppen}
   \begin{definition}
      Eine \textit{Coxetergruppe} ist ein Paar $(W,S)$, so dass:
      \begin{itemize}
         \item<2-> $S$ besteht aus Elementen von $W$ der Ordnung $2$
         \item<3-> $W = \left<S\right>$
         \item<4-> $\ell(sw) < \ell(w) \Rightarrow w = ss_2 \dots s_r$ mit $r = \ell(w)$\quad\quad\quad (Bed. \text{E})
      \end{itemize}
      \pause[5]Hierbei ist
      \[ \ell(w) := \op{min} \{ r\ |\ w = s_1 \dots s_r\} \]
      die \textit{Länge} des Elementes $w$.
   \end{definition}
\begin{itemize}
   \item<6-> $\mathfrak{H} := \{ wsw^{-1} : w \in W, s \in S\}$
   \item<7-> $m(s,t) := \op{ord}(st)$\pause[8]\quad $\Rightarrow$\quad $\underbrace{sts\dots}_{m(s,t)} = \underbrace{tst\dots}_{m(s,t)}$
\end{itemize}
\end{frame}

\begin{frame}{Hecke-Algebren}
   \begin{itemize}
      \item<1-> $R$ - komm. Ring
      \item<2-> $(W,S)$ - Coxetergruppe
      \item<3-> $a_s, b_s \in R$ für $s \in S$; \pause[4] $s,t$ konjugiert in $W$ $\Rightarrow$ $a_s = a_t,\ b_s = b_t$
   \end{itemize}
   \pause[5]\begin{definition}
      Die \textit{Hecke-Algebra} $\mathcal{H}$ von $(W,S)$ \pause über $R$\pause , zu den Parametern $(a_s)_s$, $(b_s)_s$\pause , ist der $R$-Modul
      \[ \mathcal{H} := \bigoplus_{w \in W} R T_w \pause = \{ \sum_{w \in W} c_w T_w : c_w \in R\} \]
      \pause versehen mit dem Produkt
      \begin{align*}
         T_w T_{w'} & = T_{ww'} && \text{falls $\ell(w)+\ell(w') = \ell(ww')$} \\
         \uncover<11->{T_s^2 & = a_s + b_s T_s && \text{($s \in S$)}}
      \end{align*}
   \end{definition}
\end{frame}

\begin{frame}{Das Leitbeispiel}
   Sei $a_s \equiv 1, b_s \equiv 0$: \pause $\Rightarrow T_s^2 = 1$
   \pause\begin{lemma}
      \begin{center}$T_w T_{w'} = T_{ww'} \quad \text{ für alle } w,w' \in W$\end{center}
   \end{lemma}
   \pause \begin{proof}
      \begin{itemize}
         \item<4-> O.B.d.A. $w = s$ und $\ell(sw') < \ell(w')$
         \item<5-> (Bed. E) $\Rightarrow$ $w' = s s_2 \ldots s_r$, $r = \ell(w')$
         \item<6-> $T_{w'} = T_{ss_2\ldots s_r} \pause[7] = T_s T_{s_2 \ldots s_r} \pause[8] = T_s T_v$
         \item<9-> $T_s T_{w'} = T_s T_s T_v \pause[10] = T_v \pause[11] = T_{sw'}$
      \end{itemize}
   \end{proof}
   \pause[12] $\Rightarrow \mathcal{H}$ ist nichts weiter als die Gruppenalgebra $R[W]$!
\end{frame}

\begin{frame}{Der Leitspruch}
   \begin{center}\LARGE Was für die Gruppenalgebra gilt, sollte auch für Hecke-Algebren gelten!\end{center}
\end{frame}
\end{document}
